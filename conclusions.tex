%%% CONCLUSIONS AND FUTURE WORK
\chapter{CONCLUSIONS}
\label{chap:conclusions}

%%% CONCLUSIONS
\section{Conclusions}
This thesis centers around the theory and practice of community structures detection at scale and its applications to understanding the dynamics of social networks. 

In this thesis, we propose several key enhancement to the modularity-based community detection algorithms. Despite being one of the most widely used state-of-the-art community detection approaches, modularity maximization suffers from the resolution limit problem which arises due to the implicit dependence of the modularity definition on a constant (explicitly defined in the generalized modularity as a resolution parameter). In this thesis, we establish the asymptotic theoretical upper and lower bounds on the resolution parameter of generalized modularity, which is the first result connecting the resolution limit of modularity with the random graph models. We also propose a progressive agglomerative heuristic algorithm that systematically increases the resolution parameter to partition a graph recursively. The statistical hypothesis testing checks if the partition found by each branch of the recursion is significant. If it is, this branch continues splitting the current graph; otherwise, the recursion branch terminates, accepting the null hypothesis that the current subgraph is already a community. In this thesis, we also show that the appropriately assigned edge weights can improve the quality of the detected communities. We propose an edge weighting scheme which improve the state-of-the-art approaches significantly and a generative random graph model which puts a constraint on nodes' internal degree ratio to stabilize the inference of block model, by preventing the inference algorithms, like Markov chain Monte Carlo, from getting trapped in the local optima of the log-likelihood.

In the context of information propagation, the community structures play an essential role in facilitating the local spread of information because the community members are more likely to accept inputs from each other than from the outsiders. Based on the statistical survival analysis, our community-affiliated information cascade model outperforms the feature-based approaches by almost 20\% on the Global Database of Events, Language, and Tone (GDELT) dataset as measured by the F1-scores of the predictions. We parallelize the corresponding graph representation learning algorithm for both shared memory and distributed memory machines. The parallelized stochastic gradient descent algorithm is shown to scale well to 10K+ cores in the IBM Blue Gene/Q supercomputer at RPI (ranked among the fastest academic-owned IBM Blue Gene supercomputers).

Finally, we study the patterns of the polarization evolution by analyzing millions of roll-call votes in the legislative branches of the United States. We proposed an agent-based social dynamical model explaining the formation of polarization observed in real legislative voting data and identifying the tipping points of this dynamical system. The stable state getting close to either full polarization or full consensus provides early warning signals that the system is close to the bifurcation. Our dynamical model successfully explains the directions of polarization change in 28 out of the past 30 U.S. Congresses. The hidden variable in the dynamical model, called the polarization utility, is shown to correlate with critical historical events such as the civil rights movement and Super PACs.