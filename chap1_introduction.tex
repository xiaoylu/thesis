%%% INTRODUCTION AND BACKGROUND
\chapter{INTRODUCTION AND BACKGROUND}
\label{chap:intro}

%\let\thefootnote\relax\footnotetext{
%Portions of this chapter have been submitted to:
%B.~N. Granzow, A.~A. Oberai, and M.~S. Shephard,
%``An automated approach for parallel adjoint-based error
%estimation and mesh adaptation,'' submitted for publication.}

%%% INTRODUCTION
\section{Introduction}

The arrival of the information age has profoundly changed the way how people socialize. With the rise of a wide variety of online communities, the rapid spread of information is able to reach a large number of audiences and lead to emergent public responses to various social, economic and political events in a short period. How does the information sharing in social networks lead to the emergent social phenomena? Can we use scientific models to explain and predict the viral trends which are often missed by a human observer? The availability of a wealth of data can help us answer these questions and understand how information dissemination fuels emergent social phenomena.

Accumulative influences between individuals are often used to explain the adoption of information in online networks. A user is likely to accept a message from her friends and re-share it to the others who share a tie with her. Such successive re-sharing of the same message in a snowball manner results in viral information cascades. The theories of mathematical epidemiology have been established to model the spread of contagious diseases in the early 1900s~\cite{kermack1927contribution}. The widely used SIS/SIR epidemic models~\cite{allen1994some,li2002qualitative} and its' variants are shown to be successful predictive tools to model the spread of disease in the population. Although these models are widely used to explain the disease spreading processes, there exist some barriers to model the information propagation in a social network using the SIS/SIR models. First, it is unclear when the recovery from infection occurs. After an individual has accepted the information, there is no specific sign when the user explicitly discards or forgets this information. Also, social networks are shown to be extremely sparse~\cite{huberman2008social}, composed of many small groups~\cite{clauset2004finding}. In order to accurately model the information propagation, the underlying network topology should be taken into consideration as well.

In this thesis, we discuss the interplay between the information cascades and the underlying network topology. Specifically, we focus on the impact of community structures on information propagation. Community structures are widely observed in social, biological and technological networks. In the context of information propagation, the community structures play an essential role in facilitating the local spread of information because the community members are more likely to accept inputs from each other than from the outsiders. On the other hand, the community structures slow down global information diffusion due to trapping the messages in dense regions and thus preventing global penetration.

Community detection aims at discovering the division of the network nodes into clusters such that the edges inside each cluster are generally more numerous than the edges across them. Despite being one of the most widely used state-of-the-art community detection approaches, modularity maximization suffers from the resolution limit problem which arises due to the implicit dependence of the modularity definition on a constant (explicitly defined in the generalized modularity as a resolution parameter). The modularity maximization tends to merge the small, well-formed communities into one large component while splitting one large well-formed community into smaller clusters inappropriately. Some variants of the modularity function have been proposed to either resolve this problem~\cite{chen2014community,lu2018adaptive} or to enable detection of communities at different scales. A popular choice for the latter is the generalized modularity of Reichardt and Bornholdt~\cite{reichardt2006statistical}, which scales the discovered community sizes according to a simple resolution parameter.

In this thesis, we explain the resolution limit problem~\cite{fortunato2007resolution} by defining the asymptotic theoretical upper and lower bounds on the resolution parameter of generalized modularity~\cite{lu2019asymptotic} for the modularity maximization algorithm to recover community structure correctly. To the best of our knowledge, this is the first work connecting the resolution limit of modularity with the random graph models. The unified theory shows that, in generalized modularity maximization, the well-formed communities whose densities are smaller than the resolution parameter are split into multiple clusters; but the well-formed communities whose background inter-community edge densities are larger than the resolution parameter are merged into one large component. This result reveals a ``plateaus" problem that no resolution parameter that avoids such damaging splits and mergers exists when the density of any well-formed community is smaller than the background inter-community edge density among some other well-formed communities. Therefore, based on a statistical hypothesis testing framework, we propose a progressive agglomerative heuristic algorithm that systematically increases the resolution parameter to partition a graph recursively. The statistical hypothesis testing checks if the partition found by some branch at each recursion level is significant. If it is, this recursive branch continues splitting the current graph at the next level; otherwise, this recursion branch terminates, accepting the null hypothesis that the current subgraph is already a well-formed community.

We investigate the performance of several state-of-the-art community detection algorithms on real and synthetic networks. While these community detection algorithms often take unweighted graph as input, they can be extended to accept the edge weights for community detection. A novel edge weighting scheme~\cite{lu2018adaptive} is proposed to improve the quality of communities discovered by the state-of-the-art algorithms. The edge weighting scheme~\cite{lu2018adaptive} avoids the bias of modularity maximization towards merging well-formed small communities into a large one. Our proposed edge weighting scheme works in a semi-supervised fashion: a regression model penalized by the merging of small ground truth communities is trained to convert the local edge features into edge weights. The experimental results show that, in addition to modularity maximization, the five different state-of-the-art approaches are also significantly improved by the assigned edge weights, even if their optimization goals differ from maximizing the modularity.

Besides the maximization of modularity and its generalized version, an alternative approach to detect communities is the statistical inference to fit the generative graph model such as the stochastic block model to the observed network data. The standard stochastic block model is a generative model of the graph in which nodes are organized as blocks and edges are placed between nodes independently at random, with a probability determined by the block assignments of the endpoints. Hence, the nodes in the same block are statistically indistinguishable from each other. The degree-corrected stochastic block model extends the standard stochastic block model by incorporating the node degrees, which improves the performance of community detection. Yet, it does not impose any constraints on the mixing pattern of the resulting block assignments, thus the return of the traditional assortative community structures is not guaranteed. On top of the degree-corrected stochastic block model, we propose a generative random graph model which puts a constraint on nodes' internal degree ratio in the objective function to stabilize the inference of block model in sparse networks~\cite{lu2019regularized}. Thus, our model prevents inference algorithms like Markov chain Monte Carlo from getting trapped in the local optima of the log-likelihood. Unlike the modularity maximization algorithm which always attempts to find traditional assortative communities, the inference of this regularized stochastic block model controls the mixing patterns discovered in the given network. Therefore, the inference algorithm like Markov chain Monte Carlo can reliably find the desired community structures in different optimization trials.

The diversity of the communities in which the early adopters reside is a good predictor of cascade virality~\cite{saxena2015understanding}. For this reason, we formulate information cascades model~\cite{lu2017predicting} based on the community structures of the online news media. In order to predict the viral news at the early stage of spreading, each news is considered a cascade in the network formed by news media sites. Once the media site reports a piece of news, the corresponding media site nodes in the network get infected by the cascade of this news. Our model defines the probability of infection along the network edges given the propagation delay using the statistical survival analysis. It incorporates the community-level signals which reduce the complexity of the model and improves the virality prediction accuracy. The inference algorithms of our model have been successfully parallelized for both shared memory~\cite{lu2017predicting} and distributed memory machines~\cite{lu2018scalable}. On the Global Database of Events, Language, and Tone (GDELT) dataset, the F1 score of the viral news prediction produced by our approach outperforms the results of feature-based approaches by almost 20\%. Furthermore, in order to facilitate the information spread in peer-to-peer (P2P) networks, we proposed a distributed P2P overlay network construction algorithm~\cite{lu2016towards} which maintains the power-law distribution of nodes’ degree while peers dynamically join and leave the P2P network.

Community structure plays a critical role in the development of social polarization. While social influence models~\cite{jin2014misinformation,allcott2017social,garrett2009echo,stroud2010polarization,mcpherson2001birds} study different aspects of social polarization, they share some common assumptions about human social behavior~\cite{dixit2007political}, including the following: (i) individuals iteratively update their views to reach consensus with their neighbors in a social network; (ii) the tolerance of conflicting 
views is limited in social context, so frequent active disagreements usually break social ties. These assumptions indicate that the loyalty to one's group usually leads to the conformity with views of the group's majority~\cite{deutsch1955study}, and such conformity tightens social ties within the group. Therefore, community structure is critical for understanding the evolution of social polarization.

We study the patterns of the political polarization evolution by analyzing millions of roll-call votes in the legislative branches of the United States~\cite{lu2018evolution}. The agent-based social dynamical model is proposed to explain the formation of polarization observed in real legislative voting data. The derived the tipping points of the dynamical system provides an early warning signal of political polarization when the system is close to the bifurcation. Our dynamical model successfully explains the directions of polarization change in 28 out of the past 30 elected U.S. Congresses. The hidden variable in the dynamical model, called the polarization utility, correlates well with critical historical events such as the civil rights movement and Super PACs.

%%% OUTLINE
%\section{Outline}

%Chapter \ref{chap:mech} presents the 

%In Chapter \ref{chap:refine}, the automated approach presented in Chapter \ref{chap:automated} is extended to investigate two novel approaches for solving the adjoint problem.

%Finally, we investigate ... in Chapter \ref{chap:vms}.


\section{Contributions}

This thesis centers around theory and practice of community structures detection at scale and its applications in understanding the dynamics of social networks. The specific novel contributions of this thesis include:
%
\begin{enumerate}
  \item An edge weighting scheme~\cite{lu2018adaptive} that avoids the resolution limit of the modularity maximization, improving the quality of communities discovered by the state-of-that-art community detection algorithms.
  
  \item Asymptotic theoretical lower and upper bounds on the resolution parameter of generalized modularity for the modularity maximization algorithm to recover community structure correctly~\cite{lu2019asymptotic} using the random graph properties, which connecting the resolution limit of modularity with the random graph models.
  
  \item An agglomerative heuristic algorithm which recursively divides the network into subgraphs to detect communities at different scales and a statistical hypothesis testing framework ensuring the significance of the partitions at each recursion level~\cite{lu2019asymptotic}
  
  \item A regularized stochastic block model which controls the mixing pattern of the resulting community structures~\cite{lu2019regularized} that avoids getting the inference algorithms trapped in the local optima of the log-likelihood.
  
  \item A distributed algorithm for dynamic limited scale-free network construction that at each instance of its evolution adheres to the desired Power Law of node degrees~\cite{lu2016towards} so that the network topology optimizes information propagation speed while being robust against failures.
  
  \item A model of information cascades~\cite{lu2017predicting} based on survival analysis and community detection to improve the prediction of the viral information cascades.
  
  \item A parallelized stochastic gradient descent algorithm for graph representation learning~\cite{lu2018scalable} which scales well on both shared memory and distributed memory machines.
  
  \item A dynamical model~\cite{lu2018evolution} quantifying the evolution of polarization in the U.S. Congresses elected in the past six decades. It successfully predicts the direction of polarization changes in 28 out of 30 elected U.S. Congresses. The hidden model parameter, polarization utility, correlates well with significant political or legislative changes happening at the same time.
  
\end{enumerate}